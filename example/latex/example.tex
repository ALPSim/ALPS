\documentclass[prl, superscriptaddress, showpacs, twocolumn]{revtex4}
\usepackage{vistrails}

\renewcommand{\vistrailspath}{http://alps.comp-phys.org/vistrails/run_vistrails.php}
\renewcommand{\vistrailspythonpath}{vispython}

\begin{document}

\title{A paper}
\author{Matthias Troyer}
\affiliation{Theoretische Physik, ETH Zurich, 8093 Zurich, Switzerland}
\date{\today}

\begin{abstract}
This example paper shows how figures can be embedded directly from VisTrails into a LaTeX document.
\end{abstract}

\pacs{71.10.Fd}

\hyphenation{}

\maketitle

In Fig. ~\ref{green} we show the susceptibility of various quasi-one dimensional classical and quantum spin models.

\begin{figure}[h!]
\begin{center}
\vistrail[host=alps.ethz.ch,
db=tutorials,
vtid=2,
version=225,
showspreadsheetonly]{width=8cm}
\caption{Green's function for a random point.}
\label{green}
\end{center}
\end{figure}


\acknowledgments

The calculations have been performed on the Brutus  cluster at ETH Z{\"u}rich, using the ALPS library \cite{ALPS} and VisTrails  \cite{VisTrails} .

\begin{thebibliography}{9}
\bibitem{ALPS} A.F. Albuquerque {\it et al.}, J. of Magn. and Magn. Materials {\bf 310}, 1187 (2007); \url{http://alps.comp-phys.org/}
\bibitem{VisTrails}  \url{http://www.vistrails.org/}
\end{thebibliography}

\end{document}
