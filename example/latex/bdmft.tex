%\documentclass[a4]{article} 
\documentclass[prl, superscriptaddress, showpacs, twocolumn]{revtex4}
%\usepackage{a4wide}
\usepackage{vistrails}


\begin{document}

%\title{Dynamical mean-field theory for bosons}
%\title{Diagrammatic Monte Carlo method for bosonic impurity problems}
\title{Dynamical mean field solution of the Bose-Hubbard model}
\author{Peter Anders}
\affiliation{Theoretische Physik, ETH Zurich, 8093 Zurich, Switzerland}
\author{Emanuel Gull}
\affiliation{Department of Physics, Columbia University, 538 West 120th Street, New York, NY 10027, USA}
\author{Lode Pollet}
\affiliation{Department of Physics, Harvard University, Cambridge, Massachusetts 02138, USA}
\author{Matthias Troyer}
\affiliation{Theoretische Physik, ETH Zurich, 8093 Zurich, Switzerland}
\author{Philipp Werner}
\affiliation{Theoretische Physik, ETH Zurich, 8093 Zurich, Switzerland}
\date{\today}

\begin{abstract}
Generalizing the recently proposed bosonic dynamical mean field theory (B-DMFT)  for bosonic lattice models  we find stable solutions to the B-DMFT equations and remarkable agreement with exact results from lattice Monte Carlo calculations. To solve the B-DMFT equations we present a  continuous-time Monte Carlo method for bosonic impurity models based on a diagrammatic expansion in the hybridization and condensate coupling. Applying the method
to the three-dimensional Bose Hubbard model we obtain phase boundaries with less than $2$\% deviation from the exact result, which makes B-DMFT an ideal method to study the phase diagram of bosonic optical lattice systems. The Monte Carlo solver is readily generalized to bosonic mixtures, spinful bosons, and Bose-Fermi mixtures.
\end{abstract}

\pacs{71.10.Fd}

\hyphenation{}

\maketitle


We consider a model of spinless bosons on a 3d cubic lattice. The hamiltonian of the system is given by

\begin{equation}
H = - t \sum_{\langle i,j\rangle} b_i^{\dagger}b_j +\frac{U}{2}\sum_i n_i (n_i - 1) - \mu \sum_i n_i,
\label{hamiltonian}
\end{equation}
where $t$ denotes the hopping amplitude, $U$ the on-site interaction and $\mu$ the chemical potential. 

In Fig. ~\ref{green}) we show the Green's function for one random point.

\begin{figure}[h!]
\begin{center}
\vistrail[host=alps.ethz.ch,
db=vistrails,
vtid=3,
version=220,
tag=nice figure,
showspreadsheetonly]{}
\caption{Green's function for a random point.}
\label{green}
\end{center}
\end{figure}


\acknowledgments

The calculations have been performed on the Brutus  cluster at ETH Z{\"u}rich, using the ALPS library \cite{ALPS} and VisTrails  \cite{VisTrails} .

\begin{thebibliography}{9}
\bibitem{ALPS} A.F. Albuquerque {\it et al.}, J. of Magn. and Magn. Materials {\bf 310}, 1187 (2007); \url{http://alps.comp-phys.org/}
\bibitem{VisTrails}  \url{http://www.vistrails.org/}
\end{thebibliography}

\end{document}
