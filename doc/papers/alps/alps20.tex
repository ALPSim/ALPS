%\documentclass[3p,twocolumn]{elsarticle}
\documentclass[12pt]{iopart}

\usepackage{vistrails}
%\usepackage{amsmath}
%\usepackage{amssymb}
%\usepackage{amsmath}
\usepackage{color}
\usepackage{hyperref}
\bibliographystyle{unsrt} %bibtex style according to iop latex guidelines
%\usepackage{harvard}
%\bibliographystyle{jphysicsB} %bibtex style according to iop latex guidelines

\renewcommand{\vistrailspath}{http://alps.comp-phys.org/vistrails/run_vistrails.php}
\renewcommand{\vistrailsdownload}{http://alps.comp-phys.org/vistrails/download.php}



\begin{document}

\title{The ALPS project release 2.0: \\ open source software for strongly correlated systems}


\newcounter{affiliation}
% with email addresses
%\newcommand{\myauthor}[3]{#2 {\small (#3)}$^{#1}$}
% without them
\newcommand{\myauthor}[3]{#2$^{#1}$}
\newcommand{\myaddress}[2]{\address{\refstepcounter{affiliation} $^{\arabic{affiliation}}$#2 \label{#1}}}

\author{
	\myauthor{\ref{eth}}{B. Bauer}{bauerb@phys.ethz.ch}
	\myauthor{\ref{colorado}}{L. Carr}{lcarr@mines.edu}
	\myauthor{\ref{wyoming}}{A. Feiguin}{afeiguin@uwyo.edu}
	\myauthor{\ref{utah}}{J. Freire}{juliana@cs.utah.edu}
	\myauthor{\ref{goettingen}}{S. Fuchs}{fuchs@theorie.physik.uni-goettingen.de}
	\myauthor{\ref{eth}}{L. Gamper}{gamperl@gmail.com}
	\myauthor{\ref{eth}}{J. Gukelberger}{gukelberger@phys.ethz.ch}
	\myauthor{\ref{columbia}}{E. Gull}{gull@phys.columbia.edu}
	\myauthor{\ref{bonn}}{S.~Guertler}{guertler@th.physik.uni-bonn.de}
	\myauthor{\ref{eth}}{A. Hehn}{hehn@phys.ethz.ch}
	\myauthor{\ref{jaea},\ref{crest}}{R.~Igarashi}{rigarash@hosi.phys.s.u-tokyo.ac.jp}
	\myauthor{\ref{utah}}{D. Koop}{dakoop@cs.utah.edu}
	\myauthor{\ref{eth}}{P.N. Ma}{pingnang@phys.ethz.ch}
	\myauthor{\ref{tokyo}}{H. Matsuo}{halm@looper.t.u-tokyo.ac.jp}
	\myauthor{\ref{eth},\ref{utah}}{P. Mates}{phillipmates@gmail.com}
	\myauthor{\ref{paris}}{O. Parcollet}{}
	\myauthor{\ref{affpol}}{G.~Pawlowski}{}
	\myauthor{\ref{harvard},\ref{eth}}{L.~Pollet}{pollet@phys.ethz.ch}
	\myauthor{\ref{goettingen}}{T. Pruschke}{pruschke@theorie.physik.uni-goettingen.de}
	\myauthor{\ref{brazil},\ref{utah}}{E.~Santos}{emanuele@sci.utah.edu}
	\myauthor{\ref{virginia}}{V.~Scarola}{scarola@vt.edu}
	\myauthor{\ref{lmu}}{U.~Schollw\"ock}{schollwoeck@lmu.de}
	\myauthor{\ref{utah}}{C.~Silva}{csilva@sci.utah.edu}
	\myauthor{\ref{eth}}{B.~Surer}{surerb@phys.ethz.ch}
	\myauthor{\ref{tokyo}}{S. Todo}{wistaria@ap.t.u-tokyo.ac.jp}
	\myauthor{\ref{stationq}}{S.~Trebst}{trebst@kitp.ucsb.edu}
	\myauthor{\ref{eth}}{M.~Troyer}{troyer@ethz.ch}
	\myauthor{\ref{colorado}}{M. Wall}{mwall@mymail.mines.edu}
	\myauthor{\ref{eth}}{P. Werner}{werner@phys.ethz.ch}
	\myauthor{\ref{rwth},\ref{stuttgart}}{S. Wessel}{wessel@phys.ethz.ch}
}

\myaddress{eth}{Theoretische Physik, ETH Zurich, 8093 Zurich, Switzerland}
\myaddress{colorado}{Department of Physics, Colorado School of Mines, Golden, CO 80401, USA}
\myaddress{wyoming}{}
\myaddress{utah}{}
\myaddress{goettingen}{Institut f\"ur Theoretische Physik, Georg-August-Universit\"{a}t G\"{o}ttingen, G\"{o}ttingen, Germany}
\myaddress{columbia}{Columbia University, New York, NY 10027, USA}
\myaddress{bonn}{Bethe Center for Theoretical Phyics, Universit\"{a}t Bonn, Bonn, Germany}
\myaddress{jaea}{Center for Computational Science \& e-Systems, Japan Atomic Energy Agency, 110-0015 Tokyo, Japan}
\myaddress{crest}{Core Research for Evolutional Science and Technology, Japan Science and Technology Agency, 332-0012 Kawaguchi, Japan}
\myaddress{harvard}{} 
\myaddress{paris}{} 
\myaddress{affpol}{Institute of Physics, A. Mickiewicz University, ul. Umultowska 85, 61-614 Poznan, Poland}
\myaddress{brazil}{} 
\myaddress{virginia}{}
\myaddress{lmu}{}
\myaddress{tokyo}{Department of Applied Physics, University of Tokyo, 113-8656 Tokyo, Japan}
\myaddress{stationq}{Microsoft Research, Station Q, University of California, Santa Barbara, CA 93106, USA}
\myaddress{rwth}{}
\myaddress{stuttgart}{Institut f\"ur Theoretische Physik III, Universit\"at Stuttgart, Pfaffenwaldring 57, D-70550 Stuttgart, Germany}

\begin{abstract}
We present release 2.0 of the ALPS (Algorithms and Libraries for Physics Simulations)
project, an international open source software project to develop
libraries and application programs for the simulation of strongly
correlated quantum lattice models such as quantum magnets, lattice
bosons, and strongly correlated fermion systems. Development is
centered on common XML and binary data formats, on libraries to
simplify and speed up code development, and on full-featured
simulation programs. The programs enable non-experts to start carrying
out numerical simulations by providing basic implementations of the
important algorithms for quantum lattice models: classical and quantum
Monte Carlo (QMC) using non-local updates, extended ensemble
simulations, exact and full diagonalization (ED), as well as the
density matrix renormalization group (DMRG) and quantum Monte Carlo solvers for dynamical mean field theory (DMFT). Major changes in release 2.0 include the use of HDF5 for binary data, evaluation tools in Python, support for Windows  operating system the use of CMake as build system and binary installation packages for Mac OS X and Windows, as well as integration with the VisTrails workflow provenance tool.
The software is available
from our web server at \url{http://alps.comp-phys.org/}.
\end{abstract}

\section{Introduction}
\label{}

In this paper we present release 2.0 of the ALPS project  (Algorithms and Libraries for Physics Simulations), an open source software development project for strongly correlated lattice models. We will present a short overview  and focus on new features compared to the previous releases \cite{ALPS1.2,ALPS1.3}.

\section{Building and installing ALPS}
\subsection{Build system}
\subsection{Binary installation packages}
\section{Data formats}
\section{Evaluation tools in Python}
\section{New and updated applications}
\subsection{Updated directed loop application {\tt dirloop\_sse}}
\subsection{Dynamical mean field theory solvers {\tt dmft}}
Dynamical mean field theory is a method to simulate correlated electronic systems by approximating the self-energy $\Sigma(k,\omega)$ by a momentum-independent 
function $\Sigma(\omega)$. For such a ``local'' self-energy, the diagrammmatics simplifies radically and the  problem may be mapped onto an impurity problem
coupled to a self-consistently determined bath \cite{Georges96,Kotliar06}. The method may be extended to clusters, thereby rendering it controlled in practice \cite{Maier05}.
To obtain a self-consistent solution of the impurity model we provide a simple implementation of the DMFT self-consistency condition for general single-site multi-orbital problems.
For the solution of the quantum impurity problem we provide a legacy Hirsch-Fye QMC code \cite{Hirsch86} as well as two continuous-time QMC algorithms.

The interaction expansion or ``weak coupling'' algorithm \cite{Rubtsov04,Rubtsov05} expands the impurity model partition function in the interaction around the non-interacting solution. This method, historically
the first continuous-time quantum Monte Carlo impurity solver algorithm, is widely used for the simulation of cluster problems.
The complementary hybridization expansion algorithm \cite{Werner06,Werner06Kondo} expands the impurity model partition function in the impurity-bath hybridization, treating the local (impurity) hamiltonian exactly. 
The method is usually used to simulate multi-orbital models where its capability to treat general interactions is unparalleled. A detailed description of the ALPS DMFT algorithms is given in Ref.~\cite{ALPSDMFT}.

\subsection{Time evolving block decimation code {\tt tebd}}

\section{Integration with the VisTrails workflow and computational provenance tools}
\section{Future development plans}
\section{Acknowledgements}

We thank A.F. Albuquerque, F. Alet, P. Corboz, P.Dayal, A. Honecker, A. L\"auchli, M. K\"orner,  A. Kozhevnikov, S. Manmana, I.P. McCulloch, F. Michel and R.M. Noack, for their contributions to previous versions of ALPS and  for useful discussions. We thank P. Zoller for the suggestion to provide binary installers. The development of ALPS has profited from support of the Pauli Center at ETH Z\"urich, the Kavli Institute for Theoretical Physics in Santa Barbara, the Aspen Center for Physics, the Swiss National Science Foundation, and a grant from the Army Research Office with funding from the DARPA OLE program.

\begin{figure}
\begin{center}

\vistrail[host=alps.ethz.ch,
db=vistrails,
vtid=10,
version=163,
pdf,
showspreadsheetonly]{width=8cm}
\caption{A figure produced by an ALPS VisTrails workflow: the uniform susceptibility of the Heisenberg chain and ladder. Clicking the figure retrieves the workflow used to create it. Opening that workflow lets the reader execute the full calculation.}
\label{figure}
\end{center}
\end{figure}


\begin{figure}
\begin{center}
\vistrail[host=alps.ethz.ch,
db=vistrails,
vtid=10,
version=163,
pdf,
showworkflow,
showspreadsheetonly]{width=8cm}
\caption{The workflow that created Fig. \ref{figure}. The workflow image has been created by VisTrails for the specific workflow used to create the exact version shown in the figure.}
\label{workflow}
\end{center}
\end{figure}



\begin{figure}
\begin{center}
\vistrail[host=alps.ethz.ch,
db=vistrails,
vtid=10,
version=163,
pdf,
showtree,
showspreadsheetonly]{width=8cm}
\caption{The version history tree of the workflow that created Fig. \ref{figure}.}
\end{center}
\end{figure}
\section*{References}

\bibliography{refs}
\end{document}
