%\documentclass[3p,twocolumn]{elsarticle}
\documentclass[12pt]{iopart}

\usepackage{vistrails}
%\usepackage{amsmath}
%\usepackage{amssymb}
%\usepackage{amsmath}
\usepackage{color}
\usepackage{hyperref}
\bibliographystyle{unsrt} %bibtex style according to iop latex guidelines
%\usepackage{harvard}
%\bibliographystyle{jphysicsB} %bibtex style according to iop latex guidelines

\renewcommand{\vistrailspath}{http://alps.comp-phys.org/vistrails/run_vistrails.php}
\renewcommand{\vistrailsdownload}{http://alps.comp-phys.org/vistrails/download.php}



\begin{document}

\title{The ALPS project release 2.0: \\ open source software for strongly correlated systems}


\newcounter{affiliation}
% with email addresses
%\newcommand{\myauthor}[3]{#2 {\small (#3)}$^{#1}$}
% without them
\newcommand{\myauthor}[3]{#2$^{#1}$}
\newcommand{\myaddress}[2]{\address{\refstepcounter{affiliation} $^{\arabic{affiliation}}$#2 \label{#1}}}

\author{
	\myauthor{\ref{eth}}{B. Bauer}{bauerb@phys.ethz.ch}
	\myauthor{\ref{colorado}}{L. Carr}{lcarr@mines.edu}
	\myauthor{\ref{wyoming}}{A. Feiguin}{afeiguin@uwyo.edu}
	\myauthor{\ref{utah}}{J. Freire}{juliana@cs.utah.edu}
	\myauthor{\ref{goettingen}}{S. Fuchs}{fuchs@theorie.physik.uni-goettingen.de}
	\myauthor{\ref{eth}}{L. Gamper}{gamperl@gmail.com}
	\myauthor{\ref{eth}}{J. Gukelberger}{gukelberger@phys.ethz.ch}
	\myauthor{\ref{columbia}}{E. Gull}{gull@phys.columbia.edu}
	\myauthor{\ref{bonn}}{S.~Guertler}{guertler@th.physik.uni-bonn.de}
	\myauthor{\ref{eth}}{A. Hehn}{hehn@phys.ethz.ch}
	\myauthor{\ref{jaea},\ref{crest}}{R.~Igarashi}{rigarash@hosi.phys.s.u-tokyo.ac.jp}
	\myauthor{\ref{utah}}{D. Koop}{dakoop@cs.utah.edu}
	\myauthor{\ref{eth}}{P.N. Ma}{pingnang@phys.ethz.ch}
	\myauthor{\ref{tokyo}}{H. Matsuo}{halm@looper.t.u-tokyo.ac.jp}
	\myauthor{\ref{eth},\ref{utah}}{P. Mates}{phillipmates@gmail.com}
	\myauthor{\ref{paris}}{O. Parcollet}{}
	\myauthor{\ref{affpol}}{G.~Pawlowski}{}
	\myauthor{\ref{harvard},\ref{eth}}{L.~Pollet}{pollet@phys.ethz.ch}
	\myauthor{\ref{goettingen}}{T. Pruschke}{pruschke@theorie.physik.uni-goettingen.de}
	\myauthor{\ref{brazil},\ref{utah}}{E.~Santos}{emanuele@sci.utah.edu}
	\myauthor{\ref{virginia}}{V.~Scarola}{scarola@vt.edu}
	\myauthor{\ref{lmu}}{U.~Schollw\"ock}{schollwoeck@lmu.de}
	\myauthor{\ref{utah}}{C.~Silva}{csilva@sci.utah.edu}
	\myauthor{\ref{eth}}{B.~Surer}{surerb@phys.ethz.ch}
	\myauthor{\ref{tokyo}}{S. Todo}{wistaria@ap.t.u-tokyo.ac.jp}
	\myauthor{\ref{stationq}}{S.~Trebst}{trebst@kitp.ucsb.edu}
	\myauthor{\ref{eth}}{M.~Troyer}{troyer@ethz.ch}
	\myauthor{\ref{colorado}}{M. Wall}{mwall@mymail.mines.edu}
	\myauthor{\ref{eth}}{P. Werner}{werner@phys.ethz.ch}
	\myauthor{\ref{rwth},\ref{stuttgart}}{S. Wessel}{wessel@phys.ethz.ch}
}

\myaddress{eth}{Theoretische Physik, ETH Zurich, 8093 Zurich, Switzerland}
\myaddress{colorado}{Department of Physics, Colorado School of Mines, Golden, CO 80401, USA}
\myaddress{wyoming}{}
\myaddress{utah}{}
\myaddress{goettingen}{Institut f\"ur Theoretische Physik, Georg-August-Universit\"{a}t G\"{o}ttingen, G\"{o}ttingen, Germany}
\myaddress{columbia}{Columbia University, New York, NY 10027, USA}
\myaddress{bonn}{Bethe Center for Theoretical Phyics, Universit\"{a}t Bonn, Bonn, Germany}
\myaddress{jaea}{Center for Computational Science \& e-Systems, Japan Atomic Energy Agency, 110-0015 Tokyo, Japan}
\myaddress{crest}{Core Research for Evolutional Science and Technology, Japan Science and Technology Agency, 332-0012 Kawaguchi, Japan}
\myaddress{harvard}{} 
\myaddress{paris}{} 
\myaddress{affpol}{Institute of Physics, A. Mickiewicz University, ul. Umultowska 85, 61-614 Poznan, Poland}
\myaddress{brazil}{} 
\myaddress{virginia}{}
\myaddress{lmu}{}
\myaddress{tokyo}{Department of Applied Physics, University of Tokyo, 113-8656 Tokyo, Japan}
\myaddress{stationq}{Microsoft Research, Station Q, University of California, Santa Barbara, CA 93106, USA}
\myaddress{rwth}{}
\myaddress{stuttgart}{Institut f\"ur Theoretische Physik III, Universit\"at Stuttgart, Pfaffenwaldring 57, D-70550 Stuttgart, Germany}

\begin{abstract}
We present release 2.0 of the ALPS (Algorithms and Libraries for Physics Simulations)
project, an international open source software project to develop
libraries and application programs for the simulation of strongly
correlated quantum lattice models such as quantum magnets, lattice
bosons, and strongly correlated fermion systems. Development is
centered on common XML and binary data formats, on libraries to
simplify and speed up code development, and on full-featured
simulation programs. The programs enable non-experts to start carrying
out numerical simulations by providing basic implementations of the
important algorithms for quantum lattice models: classical and quantum
Monte Carlo (QMC) using non-local updates, extended ensemble
simulations, exact and full diagonalization (ED), as well as the
density matrix renormalization group (DMRG) both in a static version and a dynamic time-evolving block decimation (TEBD) code, and quantum Monte Carlo solvers for dynamical mean field theory (DMFT). Major changes in release 2.0 include the use of HDF5 for binary data, evaluation tools in Python, support for Windows  operating system the use of CMake as build system and binary installation packages for Mac OS X and Windows, as well as integration with the VisTrails workflow provenance tool.
The software is available from our web server at \url{http://alps.comp-phys.org/}.
\end{abstract}

\section{Introduction}
\label{}

In this paper we present release 2.0 of the ALPS project  (Algorithms and Libraries for Physics Simulations), an open source software development project for strongly correlated lattice models. We will present a short overview  and focus on new features compared to the previous releases \cite{ALPS1.2,ALPS1.3}.

Quantum fluctuations and competing interactions in quantum many body
systems lead to unusual and exciting properties of strongly correlated
materials such as quantum magnetism, high temperature
superconductivity, heavy fermion
behavior, and topological quantum order.
The same strong interactions make accurate analytical treatments hard and 
direct numerical simulations are essential to increase our understanding of the unusual
properties of these systems. 

The last two decades have seen tremendous progress in the development of
algorithms.  Speedups of many orders of magnitude have been 
achieved \cite{Evertz03,Troyer03,White1992,Schollwock2005,vidal1,vidal2,Daley2004,White2004,Rubtsov04,Rubtsov05,Werner06,Werner06Kondo, Gull08_ctaux}. These
advances come at the cost of substantially increased algorithmic
complexity and challenge the current model of program development in
this research field. In contrast to other research areas, in which
large ``community codes'' are being used, the field of strongly
correlated systems has so far been based mostly on single codes developed by
individual researchers for particular projects. While the simple
algorithms used a decade ago could be easily programmed by a beginning
graduate student in a matter of weeks, it now takes substantially
longer to master and implement the new algorithms.  Thus, their use
has increasingly become restricted to a small number of experts.

The ALPS project aims to
overcome the problems posed by the growing complexity of algorithms
and the specialization of researchers onto single algorithms through
an open-source software development initiative. Its goals are to simplify the development of new codes by providing libraries and evaluation tools, and to provide ``black box'' codes of some of the most popular algorithms. To achieve these goals the ALPS project provides:
\begin{itemize}
\item {\bf standardized file formats} to simplify exchange,
distribution and archiving of simulation results and to achieve
interoperability between codes.
\item {\bf evaluation tools} for reading, post processing and writing simulation results and preparing plots.

\item {\bf generic and optimized libraries} for common aspects of
simulations of quantum and classical lattice models, to simplify code
development.
\item a set of {\bf applications} covering the major algorithms.
\item{\bf license} conditions that encourage researchers to contribute
to the ALPS project by gaining scientific credit for use of their
work.
\item {\bf outreach} through a web page \cite{alps}, mailing lists and
workshops to distribute the results and to educate researchers both
about the algorithms and the use of the applications.
\item {\bf improved reproducibility} of numerical results by
publishing source codes used to obtain published results and by integration with the VisTrails \cite{vistrails} provenance enabled workflow system.
\end{itemize}

The ready-to-use applications are useful both for 
{\it theoreticians} who want to test theoretical ideas about quantum
lattice models and to explore their properties, as well as for 
{\it experimentalists} trying to fit experimental data to theoretical
models to obtain information about the microscopic properties of
materials.

 In the following, we present a quick review of these
 aspects of the ALPS project, focusing on new features in release 2.0:
 
 \begin{itemize}
\item CMake build system \cite{cmake}, simplifying configuration and support for ALPS on  Windows
\item Binary installer packages
\item Binary file format HDF-5 \cite{hdf5} uses less space and much faster I/O
\item Python-based evaluation and plotting tools are simpler, much more flexible and powerful
\item New and revised applications: a revised version of the directed loop quantum Monte Carlo (QMC) code, QMC solvers for dynamical mean field theory (DMFT) and a time-evolving block-decimation (TEBD) algorithm for dynamics.
\item Integration with VisTrails workflow provenance system \cite{vistrails}
\item An expanded set of tutorials.
 \end{itemize}
 
 
\section{Building and installing ALPS}
\subsection{Build system}
One of the main new features in ALPS 2.0 is the change of the build system to CMake \cite{cmake}. CMake is more flexible and portable than the autotools used in ALPS version 1.3. CMake includes a graphical user interface to let the user choose installation options and manually override the paths to needed libraries if they have not been found automatically. Another advantage is that CMake enables ALPS to be built on Windows, which has been an often requested feature. A snapshot of the build instructions at the time of the release are included in the source distribution. Updated instructions will always be available on the ALPS web page.

Besides the ALPS sources the following tools and libraries are required to build ALPS:

\begin{itemize}
\item CMake build system \cite{cmake}
\item BLAS \cite{blasnetlib} and LAPACK libraries \cite{lapack}. Ideally optimized versions for the target architecture should be used.
\item The Boost C++ libraries version (included in one version of the source tarball) \cite{boost}
\item The HDF-5 library version 1.8 \cite{hdf5}
\end{itemize}

To build optional parts of ALPS it is recommended to install in addition
\begin{itemize}
\item Python version 2.5 or 2.6 \cite{python}
\item The numpy \cite{numpy}, scipy \cite{scipy} and matplotlib \cite{matplotlib} Python packages 
\item lpsolve version 4.0 \cite{lpsolve}
\item The VisTrails scientific workflow and provenance management system \cite{vistrails} and all its dependencies.
\end{itemize}





\subsection{Binary installation packages}
\subsection{LiveALPS.}

For Linux we provide an ISO disk image, called liveALPS, of a full Linux distribution with ALPS preinstalled.  It is a Linux distribution 
based on remastered CD-release of Knoppix \cite{knoppix} and is ready to use 
on a PC without special configuration. While it can burned to a DVD we recommend using it with a USB flash drive. for performance reasons.
This  solution is especially useful for Linux users that want to try ALPS, or in summers schools or lectures.

\section{Data formats}
\section{Evaluation tools in Python}
\section{New and updated applications}
\subsection{Updated directed loop application {\tt dirloop\_sse}}
\subsection{Dynamical mean field theory solvers {\tt dmft}}
Dynamical mean field theory is a method to simulate correlated electronic systems by approximating the self-energy $\Sigma(k,\omega)$ by a momentum-independent 
function $\Sigma(\omega)$. For such a ``local'' self-energy, the diagrammmatics simplifies radically and the  problem may be mapped onto an impurity problem
coupled to a self-consistently determined bath \cite{Georges96,Kotliar06}. The method may be extended to clusters, thereby rendering it controlled in practice \cite{Maier05}.
To obtain a self-consistent solution of the impurity model we provide a simple implementation of the DMFT self-consistency condition for general single-site multi-orbital problems.
For the solution of the quantum impurity problem we provide a legacy Hirsch-Fye QMC code \cite{Hirsch86} as well as two continuous-time QMC algorithms.

The interaction expansion or ``weak coupling'' algorithm \cite{Rubtsov04,Rubtsov05} expands the impurity model partition function in the interaction around the non-interacting solution. This method, historically
the first continuous-time quantum Monte Carlo impurity solver algorithm, is widely used for the simulation of cluster problems.
The complementary hybridization expansion algorithm \cite{Werner06,Werner06Kondo} expands the impurity model partition function in the impurity-bath hybridization, treating the local (impurity) hamiltonian exactly. 
The method is usually used to simulate multi-orbital models where its capability to treat general interactions is unparalleled. A detailed description of the ALPS DMFT algorithms is given in Ref.~\cite{ALPSDMFT}.

\subsection{Time evolving block decimation code {\tt tebd}}
Release 2.0 contains interfaces to the Open Source TEBD project\cite{ostebd}, which is a collection of Fortran libraries using the Time-Evolving Block Decimation (TEBD) algorithm\cite{vidal1, vidal2} to simulate time evolution of one-dimensional quantum systems.  The TEBD routines included in ALPS are an updated version of the v2.0 release of Open Source TEBD with improvements for speed and numerical stability.  TEBD can also find ground states via imaginary time evolution.  Because TEBD produces wavefunctions, a wide array of observables can be computed including local quantities, two-point correlation functions, entanglement measures, and overlaps between the state at different times.

At present, the TEBD routines can simulate the spin, boson Hubbard, hardcore boson, spinless fermions, and fermion Hubbard models from the ALPS models library.  All Hamiltonian parameters are assumed uniform throughout the system.  The measures calculated are the $z$ and $x$ magnetization, their squares, and the $\langle \hat{S}^z_i \hat{S}^z_j\rangle$ and $\langle \hat{S}^x_i \hat{S}^x_j\rangle$ correlations for the spin model; the number, its square, and the $\langle \hat{n}_i \hat{n}_j\rangle$ and $\langle \hat{a}_i^{\dagger} \hat{a}_j\rangle$ correlation functions for the boson Hubbard, hardcore boson, spinless ferimons, and fermion Hubbard models; and, additionally, the magnetization and $\langle \hat{S}^z_i \hat{S}^z_j\rangle$ correlation function for the fermion Hubbard model.  All models calculate the energy, von Neumann entanglement entropy of each site, von Neumann entanglement entropy of each bond, and the overlap of the wavefunction at time $t$ with its $t=0$ value.

In the future we plan to develop C++ libraries for performing real time evolution using the more flexible formalism of variational Matrix Product States.  These libraries will allow for more general specification of models, lattices, and observables, and improve the speed of simulations. 


\section{Integration with the VisTrails workflow and computational provenance tools}
\subsection{Computational provenance}
\subsection{VisTrails}
\subsection{CrowdLabs}
also mention VisMashups
\section{Future development plans}
\section{Acknowledgements}

We thank A.F. Albuquerque, F. Alet, P. Corboz, P.Dayal, A. Honecker, A. L\"auchli, M. K\"orner,  A. Kozhevnikov, S. Manmana, I.P. McCulloch, F. Michel and R.M. Noack, for their contributions to previous versions of ALPS and  for useful discussions. We thank P. Zoller for the suggestion to provide binary installers. The development of ALPS has profited from support of the Pauli Center at ETH Z\"urich, the Kavli Institute for Theoretical Physics in Santa Barbara, the Aspen Center for Physics, the Swiss National Science Foundation, and a grant from the Army Research Office with funding from the DARPA OLE program.

\begin{figure}
\begin{center}

\vistrail[host=alps.ethz.ch,
db=vistrails,
vtid=10,
version=163,
pdf,
showspreadsheetonly]{width=8cm}
\caption{A figure produced by an ALPS VisTrails workflow: the uniform susceptibility of the Heisenberg chain and ladder. Clicking the figure retrieves the workflow used to create it. Opening that workflow lets the reader execute the full calculation.}
\label{figure}
\end{center}
\end{figure}


\begin{figure}
\begin{center}
\vistrail[host=alps.ethz.ch,
db=vistrails,
vtid=10,
version=163,
pdf,
showworkflow,
showspreadsheetonly]{width=8cm}
\caption{The workflow that created Fig. \ref{figure}. The workflow image has been created by VisTrails for the specific workflow used to create the exact version shown in the figure.}
\label{workflow}
\end{center}
\end{figure}



\begin{figure}
\begin{center}
\vistrail[host=alps.ethz.ch,
db=vistrails,
vtid=10,
version=163,
pdf,
showtree,
showspreadsheetonly]{width=8cm}
\caption{The version history tree of the workflow that created Fig. \ref{figure}.}
\end{center}
\end{figure}
\section*{References}

\bibliography{refs}
\end{document}
